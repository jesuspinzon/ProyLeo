%Tipo de Documento [Conferencia]

\documentclass[conference]{IEEEtran}\usepackage[]{graphicx}\usepackage[]{color}
%% maxwidth is the original width if it is less than linewidth
%% otherwise use linewidth (to make sure the graphics do not exceed the margin)
\makeatletter
\def\maxwidth{ %
  \ifdim\Gin@nat@width>\linewidth
    \linewidth
  \else
    \Gin@nat@width
  \fi
}
\makeatother

\definecolor{fgcolor}{rgb}{0.345, 0.345, 0.345}
\newcommand{\hlnum}[1]{\textcolor[rgb]{0.686,0.059,0.569}{#1}}%
\newcommand{\hlstr}[1]{\textcolor[rgb]{0.192,0.494,0.8}{#1}}%
\newcommand{\hlcom}[1]{\textcolor[rgb]{0.678,0.584,0.686}{\textit{#1}}}%
\newcommand{\hlopt}[1]{\textcolor[rgb]{0,0,0}{#1}}%
\newcommand{\hlstd}[1]{\textcolor[rgb]{0.345,0.345,0.345}{#1}}%
\newcommand{\hlkwa}[1]{\textcolor[rgb]{0.161,0.373,0.58}{\textbf{#1}}}%
\newcommand{\hlkwb}[1]{\textcolor[rgb]{0.69,0.353,0.396}{#1}}%
\newcommand{\hlkwc}[1]{\textcolor[rgb]{0.333,0.667,0.333}{#1}}%
\newcommand{\hlkwd}[1]{\textcolor[rgb]{0.737,0.353,0.396}{\textbf{#1}}}%

\usepackage{framed}
\makeatletter
\newenvironment{kframe}{%
 \def\at@end@of@kframe{}%
 \ifinner\ifhmode%
  \def\at@end@of@kframe{\end{minipage}}%
  \begin{minipage}{\columnwidth}%
 \fi\fi%
 \def\FrameCommand##1{\hskip\@totalleftmargin \hskip-\fboxsep
 \colorbox{shadecolor}{##1}\hskip-\fboxsep
     % There is no \\@totalrightmargin, so:
     \hskip-\linewidth \hskip-\@totalleftmargin \hskip\columnwidth}%
 \MakeFramed {\advance\hsize-\width
   \@totalleftmargin\z@ \linewidth\hsize
   \@setminipage}}%
 {\par\unskip\endMakeFramed%
 \at@end@of@kframe}
\makeatother

\definecolor{shadecolor}{rgb}{.97, .97, .97}
\definecolor{messagecolor}{rgb}{0, 0, 0}
\definecolor{warningcolor}{rgb}{1, 0, 1}
\definecolor{errorcolor}{rgb}{1, 0, 0}
\newenvironment{knitrout}{}{} % an empty environment to be redefined in TeX

\usepackage{alltt}

%BIBLIOTECAS

% Este paquete se utiliza para generar texto o graficas de relleno.
%\usepackage{blindtext, graphicx}
%Biblioteca para graficas
\usepackage{graphicx}
%Biblioteca para lectura de caracteres ortográficos (tildes..etc. ) 
\usepackage[utf8]{inputenc}
%Biblioteca para enumeración de imagenes
\usepackage{float}
%Biblioteca para graficos vectrizados.svg 
\usepackage{svg}
\usepackage{enumerate}
%Biblioteca para enumerar figuras tablas.. etc en español 
\usepackage[spanish, es-tabla]{babel}
\usepackage[spanish]{babel}
%\usepackage[spanish,USenglish]{babel}

%INICIO DEL DOCUMENTO
\IfFileExists{upquote.sty}{\usepackage{upquote}}{}
\begin{document}
	
	
	% TITULO DEL PAPER
\title{Modelo predictivo para estimar al a\~no 2020 la poblaci\'on hispana en la ciudades de Estados Unidos}

% NOMBRE DE LOS AUTORES
\author{
	\IEEEauthorblockN{Leonel Muñoz Cedano}
	\IEEEauthorblockA{Ingeniero de Sistemas\\ 
		Universidad Distrital Francisco José de Caldas\\
		Bogotá D.C., Colombia\\
		Email: leoneling@gmail.com}
	%\and
}
%TITULO
\maketitle

%abstract del documento
%Iniciar Abstract
\begin{abstract}
	Los modelos predictivos hoy en día a nivel mundial deben ser parte
	fundamental en el desarrollo y crecimiento de las organizaciones; 
	sin tener en cuenta el tipo de actividad que realizan, ya que 
	a través de estos modelos se puenden extraer patrones 
	de los datos históricos y transaccionales con el objetivo 
	de identificar riesgos y oportunidades de negocio. 
	En ese sentido, se realizó un análisis minucioso de un Dataset 
	obtenido en Pew Research Center’s Hispanic Trends Project, con el fin
	de plantear un modelo predictivo que permita estimar la población 
	hispana en las ciudades de Estados Unidos.
\end{abstract}

%Iniciar Palabras Clave Formato IEEE
\begin{IEEEkeywords}
	Big Data, Data Mining, Dataset, Modelo predictivo, SMART.
\end{IEEEkeywords}

%Introduccion 
%SECCIÓN 1. INTRODUCCIÓN 
\section{Introducción}
	El análisis predictivo agrupa una variedad de técnicas estadísticas de modelización, aprendizaje automático y minería de datos que analiza los datos  actuales e históricos reales para hacer predicciones acerca del futuro o  acontecimientos no conocidos.\\
	%El análisis de datos siempre ha jugado un papel de vital importancia en la historia de la humanidad ya sea para comprender la naturaleza, mejorar la calidad de vida, el desarrollo de la economia, entre otras.\\
	%Además, la evolución de la tecnología ha representado un aumento considerable en cuanto a la capacidad de almacenamiento y procesamiento de información; lo cual permite el uso y tratamiento de grandes volumenes de datos.\\
	%La aplicacion del analisis de datos es infinita, puesto que todo aquello que puede ser clasificado y medido se puede analizar, por ejemplo el valor de la moneda frente a otros mercados, las visitas a un sitio web, el uso de alguna herramienta, la inteligencia de negocios, el analisis de ADN, etc.
	%Entre las diversas tecnicas para dicho analisis se pueden destacar la estadistica, el calculo de probabilidades, la mineria de datos, el big data, entre otros.\\
	Por ello se utilizaron conceptos relacionados con estadistica y ``Bigdata'' para obtener informacion relevante sobre el conjunto de datos.


%Metodologia
%SECCIÓN 3. METODOLOGIA
\section{METODOLOGIA}

 Para el desarrollo de este documento; se utilizará algunas tareas de la metogología de análisis de BigData[1].
 
  \subsection{Reconocimiento de la información}
 
  \textbf{Identificar el dominio}: Se explorará el DataSet obtenido de Pew Research Center’s Hispanic Trends Project[2], en el cual se encuentran 12544 observaciones y diferentes variables de información, entre ellas la cantidad de ciudadanos hispanos, no hispanos y total de población que se ha encontrado en algunas ciudades de los Estados Unidos y como ha sido el comportamiento de los datos en los diferentes años de la muestra (1990, 2000, 2010, 2011).\\
   
  \textbf{Variables del DataSet:} Las variables que se identificaron en el conjunto de datos son las siguientes:    	 
   	\begin{itemize}
   	\item \textbf{COUNTY}: Ciudad de un estado.%\\ 
    \item \textbf{STATE}: Estado de EEUU.%\\
   	\item \textbf{TP}: Total de población.%\\
   	\item \textbf{TPNH}: Total de población no Hispana.%\\
   	\item \textbf{TPH}: Total de población Hispana.%\\
   	\item \textbf{PPH}: Porcentaje de población Hispana.%\\
   	\item \textbf{AP}: Año de la población.%\\
   \end{itemize}
   
   \textbf{Identificar un problema}: El crecimiento poblacional hispano que ha tenido EEUU en los últimos años[2] es muy considerable; y debido al gran impacto socio-económico que esto puede acarrear en un futuro, se hace necesario poder estimar el crecimiento poblacional hispano en las diferentes ciudades principales de los EEUU. Con el desarrollo de esta investigación se propondrá un modelo predictivo que ayudará a solventar esta problematica.\\ 
    
   \textbf{Objetivos SMART}: Los objetivos del proyecto de investigación deben ser orientados con las caracteristicas SMART[3], lo que significa que estos objetivos han de contemplar las siguientes cualidades:%\\
   \begin{itemize}
     \item Specific  (Específico): Dirigirse a un área específica de mejora.%\\
     \item Measurable (Medible): Cuantificar o al menos sugerir un indicador de progreso.%\\
     \item Attainable (Alcanzable): Identificar que tipo de habilidades, actitudes u otro tipo de recursos necesitamos para cumplirlas.%\\
     \item Realistic (Realista): Los resultados esperados son acordes con los rescursos disponibles.%\\
     \item Time-bound (Oportuno): Especificar un marco de tiempo para lograr el resultado.%\\               
	\end{itemize}

  \subsection{Preguntas de investigación}
  Las preguntas de investigación[4] que se desarrollarán en el proyecto están enmarcadas en los siguientes ámbitos:
  \begin{itemize}
   \item Descriptivas: Una pregunta descriptiva es la que busca resumir una característica de un conjunto de datos.
   \item Exploratorias: Las preguntas de caracter exploratorio consisten en la busqueda de patrones o relaciones que soporten una pregunta de investigación.
   \item Inferenciales: Una pregunta inferencial consiste en el planteamiento de una hipotesis que podria ser resuelta con el analisis respectivo de la informacion.
   \item Predictivas: Las preguntas de caracter predictivo permiten analizar el comportamiento de la informacion a traves del tiempo, con el objetivo de descubrir, proyectar, o realizar hipotesis sobre estados futuros.
  \end{itemize}
  
  
  \subsection{Análisis exploratorio de los datos}
  
  El análisis exploratorio de los datos son básicamente aquellas funciones estadisticas que permiten visualizar el comportamiento de las observaciones en el DataSet en un proceso de investigación. Las funciones a utilizar son las siguientes: 
  \begin{itemize}
  	%alzate
  	\item Experimento Aleatorio[5]; Es un proceso de observación mediante el cual se selecciona un elemento de un conjunto de posibles resultados. Un experimento aleatorio es aquel en el que él resultado no se puede predecir con anterioridad a la realización misma del experimento. 
  	
  	%alzate
   \item Frecuencia relativa[5]; Sea $A$ un subconjunto del conjunto de posibles resultados de un experimento aleatorio "llamado $\Omega$". Si repetimos $N$ veces el experimento y observamos que en $N_{A}$ de esas repeticiones se obtuvo un elemento de $A$, decimos que $f_{N}(A)=\frac{N_{A}}{N}$ es la frecuencia relativa del subconjunto $A$ en esas $N$  repeticiones del experimento.

	%canavos
    \item Medidas de tendencia central[6]; 
	   \begin{itemize}
		 	\item Media: la media de las observaciones de un experimento aleatorio $x_{1},x_{2},.....x_{n}$ es el promedio aritm\'etrico de \'estas y se denota por;
		 	$$\overline{x}=\sum_{i=1}^{n} \frac{X_{i}}{n}$$ 
		 	\item Moda: la moda de un conjunto de observaciones de un experimento aleatorio es el valor de la observaci\'on que ocurre con mayor frecuencia en el conjunto.
		 	%14
		 	\item Mediana: la mediana repreesenta el valor de la variable de posición central en un conjunto de datos ordenados de un experimento aleatorio.
		 \end{itemize}
 
	 \item Varianza[6]; La Varianza de las observaciones $x_{1},x_{2},...,x_{n}$ es en esencia, el promedio del cuadrado de las distancias entre cada observaci\'on y la media del conjunto de observaciones. Se denota por:
	 $$s^{2}=\sum_{i=1}^{n} \frac{ \left( x_{i}-\overline{x}\right)^{2}}{\left(n-1 \right) } $$ 
 
	  \item Desviaci\'on est\'andar[6]; La desviaci\'on est\'andar es la raiz cuadrada de la varianza y se denota por:
	 $$s=\sqrt{\sum_{i=1}^{n} \frac{ \left( x_{i}-\overline{x}\right)^{2}}{\left(n-1 \right) } }$$ 
	 
	 \item Cuartiles[6]; Los cuartiles son dada una serie de valores  $x_{1},x_{2},...,x_{n}$ ordenados en forma creciente, podemos pensar que su cálculo podría efectuarse:
		 \begin{itemize}
		 	\item Primer cuartil (Q1) como la mediana de la primera mitad de valores.
		 	\item Segundo cuartil (Q2) como la propia mediana de la serie de valores.
		 	\item Tercer cuartil (Q3) como la mediana de la segunda mitad de valores.
		\end{itemize}
	\end{itemize}  


%Preguntas de investigacion
%SECCIÓN 3. PREGUNTAS INVESTIGACION
\section{Preguntas de investigación}

Las preguntas de investigación juegan un papel importante para el desarrollo de una investigacion de esta indole, ya que a través de ellas se logra una mejor interpretación  y definición del problema.  Las preguntas de investigación se clasifican en varios tipos de acuerdo al análisis que se desea lograr y en este caso se van a desarrollar las siguientes:

 \subsection{Preguntas de caracter descriptivo}
 Cuando se responde las preguntas de carácter descriptivo ya se puede identificar y conocer las características iniciales del conjunto de datos. Las preguntas de caracter descriptivo son:
 
  \begin{itemize}
  \item ¿Cuál es la Media de ciudadanos en EEUU durante los años  1990, 2000, 2010, 2011?
  \item ¿Qué ciudad de EEUU tiene la mayor y menor población en el año 1990?
  \item ¿Qué ciudad de EEUU tiene la mayor y menor población en el año 2000?
  \item ¿Qué ciudad de EEUU tiene la mayor y menor población en el año 2010?
  \item ¿Qué ciudad de EEUU tiene la mayor y menor población en el año 2011? 
  \item ¿Cuál es el Promedio de ciudadanos hispanos en ciudades de EEUU en los años 1990, 2000, 2010 y 2011?
  \item ¿Cuál es la ciudad de EEUU con mayor y menor cantidad de hispanos en el año 1990?
  \item ¿Qué ciudad de EEUU tiene la mayor y menor cantidad de hispanos en el año 2000?
  \item ¿Qué ciudad de EEUU tiene la mayor y menor cantidad de hispanos en el año 2010?
  \item ¿Qué ciudad de EEUU tiene la mayor y menor cantidad de hispanos en el año 2011?
  \end{itemize}
  
  \subsection{Preguntas de caracter exploratorio}
   Las preguntas de caracter exploratorio en la investigación son las siguientes:
  \begin{itemize}
   \item ¿El total de población hispana es dependiente del total de población en una ciudad?
   \item ¿El total de población hispana es dependiente del total de población no hispana en una ciudad?
   \item ¿El total de población hispana es independiente del total de población en una ciudad?
   \item ¿El total de población hispana es independiente del total de población no hispana en una ciudad?
   \item ¿El total de población no hispana es dependiente del total de población en una ciudad?

  \end{itemize}
  \subsection{Preguntas de caracter inferencial}
   Las preguntas de caracter inferencial en la investigación son las siguientes:
  \begin{itemize}
   \item ¿El total de población hispana de una ciudad se ve afectado por el total de ciudadanos?
   \item ¿El total de población hispana de una ciudad se ve afectado por el total de ciudadanos no hispanos?
  \end{itemize}
  
  \subsection{Preguntas de caracter predictivo}
   Las preguntas de caracter predictivo en la investigación son las siguientes:
	\begin{itemize}
		\item ¿Cuál será el porcentaje de crecimiento poblacional Hispana en las ciudades de EEUU al año 2020?
	\end{itemize}
	
	% Analisis exploratorio

\section{Análisis exploratorio}
El análisis exploratorio es una proceso que se realiza previamente a la aplicacación de cualquier técnica estadística a un conjunto de datos, la cual tiene como el objetivo identificar el comportamiento de los datos a través del análisis de gráficos y estadística básica permitirá explorar las distribución de los datos e identificar características tales como: valores atípicos o outliers, concentraciones de valores, saltos o discontinuidades, forma de la distribución, etc.   



\subsection{Analisis inicial}
Lo primero que se va analizar es el comportamiento que tienen los datos en los diferentes años en las variables TP, NHP y HP; en se obtiene los siguientes resultados:
%\vspace{1mm}

% Table created by stargazer v.5.2 by Marek Hlavac, Harvard University. E-mail: hlavac at fas.harvard.edu
% Date and time: mi�, abr 06, 2016 - 07:34:20 a.m.
\begin{table}[!htbp] \centering 
  \caption{Total de la población de EEUU} 
  \label{} 
\begin{tabular}{@{\extracolsep{5pt}}lccccc} 
\\[-1.8ex]\hline 
\hline \\[-1.8ex] 
Statistic & \multicolumn{1}{c}{N} & \multicolumn{1}{c}{Mean} & \multicolumn{1}{c}{St. Dev.} & \multicolumn{1}{c}{Min} & \multicolumn{1}{c}{Max} \\ 
\hline \\[-1.8ex] 
TotalCiudadanos & 12,544 & 91,700.930 & 297,470.800 & 67 & 9,889,056 \\ 
TotalNoHispanos & 12,544 & 78,932.380 & 214,518.700 & 60 & 5,511,922 \\ 
TotalHispanos & 12,544 & 12,768.550 & 102,278.800 & 0 & 4,760,974 \\ 
\hline \\[-1.8ex] 
\end{tabular} 
\end{table} 


%\vspace{1mm}
Recuerde que el anális principal de esta investigación se enfoca en estimar el crecimiento de la población hispana en algunas ciudades de EEUU, y revisando los resultados anteriores de la variable TPH se evidencia una media muy baja la cual es un valor muy significativo teniendo encuenta los valores de máximo y mínimo de la misma.  Por tal razón se ha hace necesario a través de un gráfico poder visaualizar mejor los datos de la variable TPH.   

\begin{figure}[H]
	\centering
\begin{knitrout}
\definecolor{shadecolor}{rgb}{0.969, 0.969, 0.969}\color{fgcolor}
\includegraphics[width=\maxwidth]{figure/atipicos-1} 

\end{knitrout}
	\caption{Dotplot de la variable TPH}
\end{figure}


Posteriormente se procede a realizar un análisis del conjunto de datos más al detalle, de acuerdo con la información que se tiene de los diferentes años de la muestra.

%\vspace{-10mm}
\subsection{Analizando población hispana en el año 1990}
%Los datos más representativos se dan a continuación:

% Table created by stargazer v.5.2 by Marek Hlavac, Harvard University. E-mail: hlavac at fas.harvard.edu
% Date and time: mi�, abr 06, 2016 - 07:34:21 a.m.
\begin{table}[!htbp] \centering 
  \caption{Total de la población de EEUU en el año 1990} 
  \label{} 
\begin{tabular}{@{\extracolsep{5pt}}lccccc} 
\\[-1.8ex]\hline 
\hline \\[-1.8ex] 
Statistic & \multicolumn{1}{c}{N} & \multicolumn{1}{c}{Mean} & \multicolumn{1}{c}{St. Dev.} & \multicolumn{1}{c}{Min} & \multicolumn{1}{c}{Max} \\ 
\hline \\[-1.8ex] 
TotalCiudadanos & 3,136 & 79,300.610 & 264,006.100 & 107 & 8,863,164 \\ 
TotalNoHispanos & 3,136 & 72,172.490 & 208,127.900 & 93 & 5,511,922 \\ 
TotalHispanos & 3,136 & 7,128.126 & 71,748.130 & 0 & 3,351,242 \\ 
\hline \\[-1.8ex] 
\end{tabular} 
\end{table} 


%\vspace{-10mm}
\begin{figure}[H]
	\centering
\begin{knitrout}
\definecolor{shadecolor}{rgb}{0.969, 0.969, 0.969}\color{fgcolor}
\includegraphics[width=\maxwidth]{figure/pobHis1990-1} 

\end{knitrout}
	\caption{Población Hispana en el año 1990}
\end{figure}

%\vspace{-10mm}
\subsection{Analizando población hispana en el año 2000}
%Los datos más representativos se dan a continuación:

% Table created by stargazer v.5.2 by Marek Hlavac, Harvard University. E-mail: hlavac at fas.harvard.edu
% Date and time: mi�, abr 06, 2016 - 07:34:21 a.m.
\begin{table}[!htbp] \centering 
  \caption{Total de la población de EEUU en el año 2000} 
  \label{} 
\begin{tabular}{@{\extracolsep{5pt}}lccccc} 
\\[-1.8ex]\hline 
\hline \\[-1.8ex] 
Statistic & \multicolumn{1}{c}{N} & \multicolumn{1}{c}{Mean} & \multicolumn{1}{c}{St. Dev.} & \multicolumn{1}{c}{Min} & \multicolumn{1}{c}{Max} \\ 
\hline \\[-1.8ex] 
TotalCiudadanos & 3,136 & 89,735.040 & 292,674.700 & 67 & 9,519,338 \\ 
TotalNoHispanos & 3,136 & 78,476.900 & 214,891.300 & 60 & 5,277,125 \\ 
TotalHispanos & 3,136 & 11,258.140 & 96,312.440 & 1 & 4,242,213 \\ 
\hline \\[-1.8ex] 
\end{tabular} 
\end{table} 


%\vspace{-5mm}
\begin{figure}[H]
	\centering
\begin{knitrout}
\definecolor{shadecolor}{rgb}{0.969, 0.969, 0.969}\color{fgcolor}
\includegraphics[width=\maxwidth]{figure/pobHisp2000-1} 

\end{knitrout}
	\caption{Población Hispana en el año 2000}
\end{figure}

%\vspace{-10mm}
\subsection{Analizando población hispana en el año 2010}
%Los datos más representativos se dan a continuación:

% Table created by stargazer v.5.2 by Marek Hlavac, Harvard University. E-mail: hlavac at fas.harvard.edu
% Date and time: mi�, abr 06, 2016 - 07:34:21 a.m.
\begin{table}[!htbp] \centering 
  \caption{Total de la población de EEUU en el año 2010} 
  \label{} 
\begin{tabular}{@{\extracolsep{5pt}}lccccc} 
\\[-1.8ex]\hline 
\hline \\[-1.8ex] 
Statistic & \multicolumn{1}{c}{N} & \multicolumn{1}{c}{Mean} & \multicolumn{1}{c}{St. Dev.} & \multicolumn{1}{c}{Min} & \multicolumn{1}{c}{Max} \\ 
\hline \\[-1.8ex] 
TotalCiudadanos & 3,136 & 98,430.440 & 313,221.000 & 82 & 9,818,605 \\ 
TotalNoHispanos & 3,136 & 82,336.350 & 216,856.000 & 64 & 5,130,716 \\ 
TotalHispanos & 3,136 & 16,094.090 & 115,731.900 & 0 & 4,687,889 \\ 
\hline \\[-1.8ex] 
\end{tabular} 
\end{table} 


%\vspace{-5mm}
\begin{figure}[H]
	\centering
\begin{knitrout}
\definecolor{shadecolor}{rgb}{0.969, 0.969, 0.969}\color{fgcolor}
\includegraphics[width=\maxwidth]{figure/pobHisp2010-1} 

\end{knitrout}
	\caption{Población Hispana en el año 2010}
\end{figure}

%\vspace{-10mm}
\subsection{Analizando población hispana en el año 2011}
%Los datos más representativos se dan a continuación:

% Table created by stargazer v.5.2 by Marek Hlavac, Harvard University. E-mail: hlavac at fas.harvard.edu
% Date and time: mi�, abr 06, 2016 - 07:34:21 a.m.
\begin{table}[!htbp] \centering 
  \caption{Total de la población de EEUU en el año 2011} 
  \label{} 
\begin{tabular}{@{\extracolsep{5pt}}lccccc} 
\\[-1.8ex]\hline 
\hline \\[-1.8ex] 
Statistic & \multicolumn{1}{c}{N} & \multicolumn{1}{c}{Mean} & \multicolumn{1}{c}{St. Dev.} & \multicolumn{1}{c}{Min} & \multicolumn{1}{c}{Max} \\ 
\hline \\[-1.8ex] 
TotalCiudadanos & 3,136 & 99,337.630 & 316,723.400 & 90 & 9,889,056 \\ 
TotalNoHispanos & 3,136 & 82,743.800 & 217,998.000 & 76 & 5,128,082 \\ 
TotalHispanos & 3,136 & 16,593.830 & 118,221.400 & 0 & 4,760,974 \\ 
\hline \\[-1.8ex] 
\end{tabular} 
\end{table} 


%\vspace{-10mm}
\begin{figure}[H]
	\centering
\begin{knitrout}
\definecolor{shadecolor}{rgb}{0.969, 0.969, 0.969}\color{fgcolor}
\includegraphics[width=\maxwidth]{figure/pobHisp2011-1} 

\end{knitrout}
	\caption{Población Hispana en el año 2011}
\end{figure}

%\vspace{-10mm}
\subsection{Percentiles del conjunto de datos}
El percentil de orden \(k\) es el cuantil de orden \(\dfrac {k} {100}\). El recorrido intercuantil refleja la variabilidad de las observaciones comprendidas entre los percentiles 25 y 75 en el conjunto de datos. En esta sesión se obtienen los percentiles del 25\%,  50\% y 75\% de las variables Total Población (TP) y Total Población Hispana (TPH) en los diferentes años de la muestra.

%\vspace{4mm}
%\begin{itemize}
%\item Percentiles del población en el año 1990
% latex table generated in R 3.2.3 by xtable 1.8-2 package
% Wed Apr 06 07:34:22 2016
\begin{table}[ht]
\centering
\begin{tabular}{rrr}
  \hline
 & PercentilesTP & PercentilesTPH \\ 
  \hline
25\% & 10360.00 & 67.00 \\ 
  50\% & 22224.00 & 208.00 \\ 
  75\% & 54771.50 & 1159.00 \\ 
   \hline
\end{tabular}
\caption{Percentiles de TP y TPH en el año 1990} 
\end{table}


% latex table generated in R 3.2.3 by xtable 1.8-2 package
% Wed Apr 06 07:34:22 2016
\begin{table}[ht]
\centering
\begin{tabular}{rrr}
  \hline
 & PercentilesTP & PercentilesTPH \\ 
  \hline
25\% & 11264.25 & 155.00 \\ 
  50\% & 24658.00 & 493.00 \\ 
  75\% & 61844.25 & 2411.50 \\ 
   \hline
\end{tabular}
\caption{Percentiles de TP y TPH en el año 2000} 
\end{table}

	
% latex table generated in R 3.2.3 by xtable 1.8-2 package
% Wed Apr 06 07:34:22 2016
\begin{table}[ht]
\centering
\begin{tabular}{rrr}
  \hline
 & PercentilesTP & PercentilesTPH \\ 
  \hline
25\% & 11154.75 & 262.75 \\ 
  50\% & 25901.50 & 892.00 \\ 
  75\% & 67012.50 & 4226.25 \\ 
   \hline
\end{tabular}
\caption{Percentiles de TP y TPH en el año 2010} 
\end{table}

	
% latex table generated in R 3.2.3 by xtable 1.8-2 package
% Wed Apr 06 07:34:22 2016
\begin{table}[ht]
\centering
\begin{tabular}{rrr}
  \hline
 & PercentilesTP & PercentilesTPH \\ 
  \hline
25\% & 11145.00 & 284.00 \\ 
  50\% & 25896.00 & 926.00 \\ 
  75\% & 67398.75 & 4417.75 \\ 
   \hline
\end{tabular}
\caption{Percentiles de TP y TPH en el año 2011} 
\end{table}

	
	% Solución de preguntas

\section{Solución de Preguntas}
Ahora se procederá a responder todas las preguntas que se plantearón al inicio de la investigación.



\subsection{Caracter descriptivo}

%\begin{enumerate}
A continuación se enumeran los promedios de la variable total ploblación en cada uno de los años del DataSet:
%\item los promedios por  año de los ciudadanos en EEUU, son los siguientes\\
\begin{itemize}
\item promedio del año 1990:
[1] 79300.61

\item promedio del año 2000:
[1] 89735.04

\item promedio del año 2010:
[1] 98430.44

\item promedio del año 2011:
[1] 99337.63

\end{itemize}

%\item Ciudad con más población en 1990
% latex table generated in R 3.2.3 by xtable 1.8-2 package
% Wed Apr 06 07:34:22 2016
\begin{table}[ht]
\centering
\begin{tabular}{rllr}
  \hline
 & Ciudad & Estado & Poblacion \\ 
  \hline
1 & Los Angeles & California & 8863164 \\ 
   \hline
\end{tabular}
\caption{Ciudad con más población en el año 1990} 
\end{table}


%Ciudades con menos población en el año 1990
% latex table generated in R 3.2.3 by xtable 1.8-2 package
% Wed Apr 06 07:34:22 2016
\begin{table}[ht]
\centering
\begin{tabular}{rllr}
  \hline
 & Ciudad & Estado & Poblacion \\ 
  \hline
1 & Loving & Texas & 107 \\ 
   \hline
\end{tabular}
\caption{Ciudad con menos población en el año 1990} 
\end{table}


%\item Ciudad con más población en 2000
% latex table generated in R 3.2.3 by xtable 1.8-2 package
% Wed Apr 06 07:34:22 2016
\begin{table}[ht]
\centering
\begin{tabular}{rllr}
  \hline
 & Ciudad & Estado & Poblacion \\ 
  \hline
1 & Los Angeles & California & 8863164 \\ 
   \hline
\end{tabular}
\caption{Ciudad con más población en el año 2000} 
\end{table}


%Ciudades con menos población en el año 2000
% latex table generated in R 3.2.3 by xtable 1.8-2 package
% Wed Apr 06 07:34:22 2016
\begin{table}[ht]
\centering
\begin{tabular}{rllr}
  \hline
 & Ciudad & Estado & Poblacion \\ 
  \hline
1 & Loving & Texas &  67 \\ 
   \hline
\end{tabular}
\caption{Ciudad con menos población en el año 2000} 
\end{table}


%\item Ciudad con más población en 2010
% latex table generated in R 3.2.3 by xtable 1.8-2 package
% Wed Apr 06 07:34:22 2016
\begin{table}[ht]
\centering
\begin{tabular}{rllr}
  \hline
 & Ciudad & Estado & Poblacion \\ 
  \hline
1 & Los Angeles & California & 9818605 \\ 
   \hline
\end{tabular}
\caption{Ciudad con más población en el año 2010} 
\end{table}


%Ciudades con menos población en el año 2010
% latex table generated in R 3.2.3 by xtable 1.8-2 package
% Wed Apr 06 07:34:22 2016
\begin{table}[ht]
\centering
\begin{tabular}{rllr}
  \hline
 & Ciudad & Estado & Poblacion \\ 
  \hline
1 & Loving & Texas &  82 \\ 
   \hline
\end{tabular}
\caption{Ciudad con menos pobl. en 2010} 
\end{table}


%\item Ciudad con más población en 2011
% latex table generated in R 3.2.3 by xtable 1.8-2 package
% Wed Apr 06 07:34:22 2016
\begin{table}[ht]
\centering
\begin{tabular}{rllr}
  \hline
 & Ciudad & Estado & Poblacion \\ 
  \hline
1 & Los Angeles & California & 9889056 \\ 
   \hline
\end{tabular}
\caption{Ciudad con más población en el año 2011} 
\end{table}


%Ciudades con menos población en el año 2011
% latex table generated in R 3.2.3 by xtable 1.8-2 package
% Wed Apr 06 07:34:22 2016
\begin{table}[ht]
\centering
\begin{tabular}{rllr}
  \hline
 & Ciudad & Estado & Poblacion \\ 
  \hline
1 & Kalawao & Hawaii &  90 \\ 
   \hline
\end{tabular}
\caption{Ciudad con menos población en el año 2011} 
\end{table}


%A continuación se enumera las medias de la variables Total Población (TP), Total Población  No Hispana (TPNH) y Total Ploblación Hispana (TPH) en cada uno de los años del DataSet:
%\item los promedios por año de ciudadanos hispanos, son los siguientes\\
%\begin{itemize}
%\item promedio del año 1990:
%<<mseisa,results='asis',echo=FALSE, warning=FALSE, message=FALSE>>=
%	mean(datos1990$TPH)
%@
%\item promedio del año 2000:
%<<seisb,results='asis',echo=FALSE, warning=FALSE, message=FALSE>>=
%	mean(datos2000$TPH)
%@
%\item promedio del año 2010:
%<<seisc,results='asis',echo=FALSE, warning=FALSE, message=FALSE>>=
%	mean(datos2010$TPH)
%@
%\item promedio del año 2011:
%<<seisd,results='asis',echo=FALSE, warning=FALSE, message=FALSE>>=
%	mean(datos2011$TPH)
%@
%\end{itemize}
%	grupoMediaTP 	<- c(mean(datos1990$TP), mean(datos2000$TP), mean(datos2010$TP), mean(datos2011$TP))
%	grupoMediaTPNH 	<- c(mean(datos1990$TPNH), mean(datos2000$TPNH), mean(datos2010$TPNH),  mean(datos2011$TPNH))
%	grupoMediaTPH 	<- c(mean(datos1990$TPH), mean(datos2000$TPH), mean(datos2010$TPH), mean(datos2011$TPH))
%	mediasFull<-data.frame(TP=grupoMediaTP, TPNH=grupoMediaTPNH, TPH=grupoMediaTPH)	 	
%	xtable(mediasFull,"Valores de las medias en TP, TPHN y TPH")
A continuación se enumera las medias de la variables Total Población (TP), Total Población  No Hispana (TPNH) y Total Ploblación Hispana (TPH) en cada uno de los años del DataSet:

% latex table generated in R 3.2.3 by xtable 1.8-2 package
% Wed Apr 06 07:34:22 2016
\begin{table}[ht]
\centering
\begin{tabular}{rrrrr}
  \hline
 & Año 1990 & Año 2000 & Año 2010 & Año 2011 \\ 
  \hline
Media TP & 79300.61 & 89735.04 & 98430.44 & 99337.63 \\ 
  Media TPNH & 72172.49 & 78476.90 & 82336.35 & 82743.80 \\ 
  Media TPH & 7128.13 & 11258.14 & 16094.09 & 16593.83 \\ 
   \hline
\end{tabular}
\caption{Valores de las medias en TP, TPHN y TPH} 
\end{table}



El análisis continua con los siguientes resultados.
%Ciudad con mas población hispana en el año 1990
% latex table generated in R 3.2.3 by xtable 1.8-2 package
% Wed Apr 06 07:34:22 2016
\begin{table}[ht]
\centering
\begin{tabular}{rllr}
  \hline
 & Ciudad & Estado & PoblacionH \\ 
  \hline
1 & Los Angeles & California & 3351242 \\ 
   \hline
\end{tabular}
\caption{Ciudad con mayor TPH en el año 1990} 
\end{table}


%Ciudades con menos población hispana en el año 1990
% latex table generated in R 3.2.3 by xtable 1.8-2 package
% Wed Apr 06 07:34:22 2016
\begin{table}[ht]
\centering
\begin{tabular}{rllr}
  \hline
 & Ciudad & Estado & PoblacionH \\ 
  \hline
1 & Garfield & Montana &   0 \\ 
  2 & Petroleum & Montana &   0 \\ 
  3 & Arthur & Nebraska &   0 \\ 
  4 & Blaine & Nebraska &   0 \\ 
  5 & McPherson & Nebraska &   0 \\ 
  6 & Wheeler & Nebraska &   0 \\ 
  7 & Billings & North Dakota &   0 \\ 
  8 & Campbell & South Dakota &   0 \\ 
  9 & McPherson & South Dakota &   0 \\ 
   \hline
\end{tabular}
\caption{Ciudad con menor TPH en el año 1990} 
\end{table}


%Ciudad con más población hispana en el año 2000
% latex table generated in R 3.2.3 by xtable 1.8-2 package
% Wed Apr 06 07:34:22 2016
\begin{table}[ht]
\centering
\begin{tabular}{rllr}
  \hline
 & Ciudad & Estado & PoblacionH \\ 
  \hline
1 & Los Angeles & California & 4242213 \\ 
   \hline
\end{tabular}
\caption{Ciudad con mayor TPH en el año 2000} 
\end{table}


%Ciudades con menos población hispana en el año 2000
% latex table generated in R 3.2.3 by xtable 1.8-2 package
% Wed Apr 06 07:34:22 2016
\begin{table}[ht]
\centering
\begin{tabular}{rllr}
  \hline
 & Ciudad & Estado & PoblacionH \\ 
  \hline
1 & Blaine & Nebraska &   1 \\ 
  2 & Slope & North Dakota &   1 \\ 
   \hline
\end{tabular}
\caption{Ciudad con menor TPH en el año 2000} 
\end{table}


%Ciudad con más población hispana en el año 2010
% latex table generated in R 3.2.3 by xtable 1.8-2 package
% Wed Apr 06 07:34:22 2016
\begin{table}[ht]
\centering
\begin{tabular}{rllr}
  \hline
 & Ciudad & Estado & PoblacionH \\ 
  \hline
1 & Los Angeles & California & 4687889 \\ 
   \hline
\end{tabular}
\caption{Ciudad con mayor TPH en el año 2010} 
\end{table}


%Ciudad con menos población hispana en el año 2010
% latex table generated in R 3.2.3 by xtable 1.8-2 package
% Wed Apr 06 07:34:22 2016
\begin{table}[ht]
\centering
\begin{tabular}{rllr}
  \hline
 & Ciudad & Estado & PoblacionH \\ 
  \hline
1 & Blaine & Nebraska &   0 \\ 
   \hline
\end{tabular}
\caption{Ciudad con menor TPH en el año 2010} 
\end{table}


%Ciudad con más población hispana en el año 2011
% latex table generated in R 3.2.3 by xtable 1.8-2 package
% Wed Apr 06 07:34:22 2016
\begin{table}[ht]
\centering
\begin{tabular}{rllr}
  \hline
 & Ciudad & Estado & PoblacionH \\ 
  \hline
1 & Los Angeles & California & 4760974 \\ 
   \hline
\end{tabular}
\caption{Ciudad con mayor TPH en el año 2011} 
\end{table}


%Ciudad con menos población hispana en el año 2011
% latex table generated in R 3.2.3 by xtable 1.8-2 package
% Wed Apr 06 07:34:22 2016
\begin{table}[ht]
\centering
\begin{tabular}{rllr}
  \hline
 & Ciudad & Estado & PoblacionH \\ 
  \hline
1 & Blaine & Nebraska &   0 \\ 
   \hline
\end{tabular}
\caption{Ciudad con menor TPH en el año 2011} 
\end{table}


\subsection{Matriz de correlación}
A continuación se visualiza la matriz de correlación que existe entre la variables cuantitativas del conjunto de datos: 
%Ciudad con menos población hispana en el año 2011
% latex table generated in R 3.2.3 by xtable 1.8-2 package
% Wed Apr 06 07:34:22 2016
\begin{table}[ht]
\centering
\begin{tabular}{rrrrrr}
  \hline
 & TP & TPNH & TPH & PPH & AP \\ 
  \hline
TP & 1.00 & 0.97 & 0.87 & 0.17 & 0.03 \\ 
  TPNH & 0.97 & 1.00 & 0.73 & 0.12 & 0.02 \\ 
  TPH & 0.87 & 0.73 & 1.00 & 0.25 & 0.04 \\ 
  PPH & 0.17 & 0.12 & 0.25 & 1.00 & 0.13 \\ 
  AP & 0.03 & 0.02 & 0.04 & 0.13 & 1.00 \\ 
   \hline
\end{tabular}
\caption{Matrix de correlación} 
\end{table}

\includegraphics[width=\maxwidth]{figure/correlacion-1} 


	
	%BIBLIOGRAFÍA
	%ENTORNO {thebibliography}
	%Permite al autor listar las referencias utilizadas y citarlas en algun punto del texto.
	
	\newpage
	 \begin{thebibliography}{1}
	 	
	 	\bibitem{biblio1}
	 	S. Mohanty, M. Jagadeesh and H. Srivatsa, Big Data Imperatives: Enterprise Big Data Warehouse, BI Implementations and Analytics, Published Apress, Isbn: 978-1-4302-4872-9, New York, 2013.
	 	
	 	\bibitem{biblio2}
	 	pewhispanic.org, Pew Research Center’s Hispanic Trends Project, U.S. Hispanic Population by County, 1980-2011. Disponible en: http://www.pewhispanic.org/2013/08/29/u-s-hispanic-population-by-county-1980-2011/, 2013.
	 	
	 	\bibitem{biblio3}
	 	G. C. Canavos, Probabilidad y estadística: Aplicaciones y métodos, Virginia Commonwealth University, Published McGRAW HILL, 1988.
	 	
	 	\bibitem{biblio4}
	 	Alzate Marco, 250 Conceptos de Probabilidad, variables aleatorias y procesos estocásticos en redes de comunicaciones, Universidad Distrital Fransisco José de Caldas, pag 15-123, 2005.
	 	
	 \end{thebibliography}
	
\end{document}
