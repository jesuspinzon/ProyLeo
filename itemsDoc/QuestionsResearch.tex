%SECCIÓN 3. PREGUNTAS INVESTIGACION


\section{Preguntas de investigación}
Las preguntas de investigación juegan un papel importante para el desarrollo de una investigacion de esta indole, ya que a través de ellas se logra una mejor interpretación  y definición del problema.  Las preguntas de investigación se clasifican en varios tipos de acuerdo al análisis que se desea lograr y en este caso se van a desarrollar las siguientes:
 \subsection{Preguntas de caracter descriptivo}
 Cuando se responde las preguntas de carácter descriptivo ya se puede identificar y conocer las características iniciales del conjunto de datos. Las preguntas de caracter descriptivo son:
  \begin{itemize}
  \item ¿Cuál es la Media de ciudadanos en EEUU durante los años  1990, 2000, 2010, 2011?
  \item ¿Qué ciudad de EEUU tiene la mayor y menor población en el año 1990?
  \item ¿Qué ciudad de EEUU tiene la mayor y menor población en el año 2000?
  \item ¿Qué ciudad de EEUU tiene la mayor y menor población en el año 2010?
  \item ¿Qué ciudad de EEUU tiene la mayor y menor población en el año 2011? 
   \item ¿Cuál es el Promedio de ciudadanos hispanos en ciudades de EEUU en los años 1990, 2000, 2010 y 2011?
   \item ¿Cuál es la ciudad de EEUU con mayor y menor cantidad de hispanos en el año 1990?
   \item ¿Qué ciudad de EEUU tiene la mayor y menor cantidad de hispanos en el año 2000?
   \item ¿Qué ciudad de EEUU tiene la mayor y menor cantidad de hispanos en el año 2010?
   \item ¿Qué ciudad de EEUU tiene la mayor y menor cantidad de hispanos en el año 2011?
  

  \end{itemize}
  \subsection{Preguntas de caracter exploratorio}
   Las preguntas de caracter exploratorio consisten en la busqueda de patrones o relaciones que soporten una pregunta de investigación.
  \begin{itemize}
   \item ¿Por definir?

  \end{itemize}
  \subsection{Preguntas de caracter inferencial}
   Las preguntas de caracter inferencial consisten en el planteamiento de una hipotesis que podria ser resuelta con el analisis respectivo de la informacion
  \begin{itemize}
   \item ¿Por definir?
  \end{itemize}
  \subsection{Preguntas de caracter predictivo}
   Las preguntas de caracter predictivo permiten analizar el comportamiento de la informacion a traves del tiempo, con el objetivo de descubrir, proyectar, o realizar hipotesis sobre estados futuros.
	\begin{itemize}
		\item ¿Cuál será el porcentaje de crecimiento poblacional Hispana en las ciudades de EEUU al año 2020?
	\end{itemize}