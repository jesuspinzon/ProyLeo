%SECCIÓN 3. METODOLOGIA
\section{METODOLOGIA}

 Para el desarrollo de este documento se planteará las siguientes tareas [1].
 
  \subsection{Reconocimiento de la información}
 
  \begin{itemize}
   \item \textbf{Identificar el dominio}: Se va a explorar los datos obtenidos de la página Pew Research Center’s Hispanic Trends Project [2], en el cual se pueden visualizar la cantidad de ciudadanos hispanos que se han encontrado en varias ciudades de los Estados Unidos y como ha sido el crecimiento de los mismos en diferentes años de la muestra (1990, 2000, 2010, 2011).\\
   
   \item \textbf{Variables del DataSet:}
    	 
   	\textbf{COUNTY} : Ciudad de un estado\\ 
    \textbf{STATE} 	: Estado de USA\\
   	\textbf{TP} 	: Total de población\\
   	\textbf{TPNH} 	: Total de población no Hispana\\
   	\textbf{TPH} 	: Total de población Hispana\\
   	\textbf{PPH} 	: Porcentaje de población Hispana\\
   	\textbf{AP} 	: Año de la población\\
   	
 
    \item \textbf{Identificar un problema}: El crecimiento poblacional hispano que ha tenido EEUU en los últimos años [2] es muy considerable; y debido al gran impacto socio-económico que esto puede traer en un futuro, se hace necesario poder estimar el crecimiento poblacional hispano en las diferentes ciudades principales de los EEUU. El desarrollo de esta investigación propondrá un modelo predictivo que ayudará a solventar esta problematica.\\ 
    
   \item \textbf{Objetivos SMART}: Los objetivos del proyecto de investigación deben ser orientados con caracteristicas SMART, lo que significa que estos objetivos han de contemplar las siguientes cualidades
    \begin{itemize}
     \item Specific  (Específico)
     \item Measurable (Medible)
     \item Attainable (Alcanzable)
     \item Realistic (Realista)
     \item Time-bound (Oportuno)               
    \end{itemize}
  \end{itemize}
  \subsection{Preguntas de investigación}
  Las preguntas de investigación que se desarrollarán en el proyecto están enmarcadas en los siguientes ámbitos:
  \begin{itemize}
   \item Descriptivas
   \item Exploratorias
   \item Inferenciales
   \item Predictivas
  \end{itemize}
  \subsection{Analisis exploratorio}
  \begin{itemize}
  	%alzate
  	\item Experimento Aleatorio [3]; Es un proceso de observación mediante el cual se selecciona un elemento de un conjunto de posibles resultados. Un experimento aleatorio es aquel en el que el resultado no se puede predecir con anterioridad a la realización misma del experimento. 
  	
  	%alzate
   \item Frecuencia relativa [3]; Sea $A$ un subconjunto del conjunto de posibles resultados de un experimento aleatorio "llamado $\Omega$". Si repetimos $N$ veces el experimento y observamos que en $N_{A}$ de esas repeticiones se obtuvo un elemento de $A$, decimos que $f_{N}(A)=\frac{N_{A}}{N}$ es la frecuencia relativa del subconjunto $A$ en esas $N$  repeticiones del experimento.
%canavos
   
   \item Medidas de tendencia central[4] 
	   \begin{itemize}
		 	\item Media: la media de las observaciones de un experimento aleatorio $x_{1},x_{2},.....x_{n}$ es el promedio aritm\'etrico de \'estas y se denota por;
		 	$$\overline{x}=\sum_{i=1}^{n} \frac{X_{i}}{n}$$ 
		 	\item Moda: la moda de un conjunto de observaciones de un experimento aleatorio es elvalor de la observaci\'on que ocurre con mayor frecuencia en el conjunto.
		 	%14
		 	\item Mediana: la mediana repreesenta el valor de la variable de posición central en un conjunto de datos ordenados de un experimento aleatorio.
		 \end{itemize}
 %15\
 \item Varianza [4] : La Varianza de las observaciones $x_{1},x_{2},...,x_{n}$ es en esencia, el promedio del cuadrado de las distancias entre cada observaci\'on y la media del conjunto de observaciones. Se denota por:
 $$s^{2}=\sum_{i=1}^{n} \frac{ \left( x_{i}-\overline{x}\right)^{2}}{\left(n-1 \right) } $$ 
 
 
 \item Desviaci\'on est\'andar [4]: La desviaci\'on est\'andar es la raiz cuadrada de la varianza y se denota por:
 $$s=\sqrt{\sum_{i=1}^{n} \frac{ \left( x_{i}-\overline{x}\right)^{2}}{\left(n-1 \right) } }$$ 
 
\end{itemize}  



