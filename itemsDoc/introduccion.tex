%SECCIÓN 1. INTRODUCCIÓN 
\section{Introducción}
	El análisis predictivo agrupa una variedad de técnicas estadísticas de modelización, aprendizaje automático y minería de datos que analiza los datos  actuales e históricos reales para hacer predicciones acerca del futuro o  acontecimientos no conocidos.\\
	%El análisis de datos siempre ha jugado un papel de vital importancia en la historia de la humanidad ya sea para comprender la naturaleza, mejorar la calidad de vida, el desarrollo de la economia, entre otras.\\
	%Además, la evolución de la tecnología ha representado un aumento considerable en cuanto a la capacidad de almacenamiento y procesamiento de información; lo cual permite el uso y tratamiento de grandes volumenes de datos.\\
	%La aplicacion del analisis de datos es infinita, puesto que todo aquello que puede ser clasificado y medido se puede analizar, por ejemplo el valor de la moneda frente a otros mercados, las visitas a un sitio web, el uso de alguna herramienta, la inteligencia de negocios, el analisis de ADN, etc.
	%Entre las diversas tecnicas para dicho analisis se pueden destacar la estadistica, el calculo de probabilidades, la mineria de datos, el big data, entre otros.\\
	Por ello se utilizaron conceptos relacionados con estadistica y ``Bigdata'' para obtener informacion relevante sobre el conjunto de datos.
